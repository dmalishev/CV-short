\documentclass{article} \oddsidemargin -0,05 cm \textwidth
16.0cm \topmargin 0pt \textheight 24cm
\usepackage{color}
\usepackage{amsmath,amssymb}
\usepackage{floatflt}
%\usepackage[colorlinks]{hyperref}
\usepackage{psfrag}
\usepackage{jcappub} %
\bibliographystyle{JHEP-2}
\newcommand{\bc}{\begin{center}}
\newcommand{\be}{\begin{equation}}
\newcommand{\ec}{\end{center}}
\newcommand{\ee}{\end{equation}}
\renewcommand\refname{}
\begin{document}
\section*{ Denys Malyshev}
 \noindent
Bogolyubov Institute for Theoretical Physics, Kiev, Ukraine\\
\emph{e-mail}: dmalishev@bitp.kiev.ua

\subsection*{Scientific interests}
\begin{itemize}
\item \textbf{$\nu$MSM, Dark Matter search}: Recently by combining the data from all available for the moment XMM-NEWTON observations, Boyarsky et.al. put the tight constraints on the X-ray flux from decaying DM in X-ray energy band. For the case of $\nu$MSM sterile neutrino DM case they put corresponding constraints on mixing angle as function of sterile neutrino mass. However, XMM-NEWTON has a number of strong instrumental lines. The flux in these lines is much higher than the background one, thus any background subtraction algorithm will be limited by higher statistical errors and provide weaker constraints at the position of the lines. Due to this the obtained constraint on mixing angle is not a powerlaw-like function of neutrino mass, but has the number of features which correspond to the energies of instrumental lines. The obtained constraints are much weaker at these positions. Unfortunately, XMM-NEWTON has quite large number of instrumental lines and the energy of the line of DM decay could be similar to one of these instrumental lines and hide under the corresponding weaker constraints.

    The windows with weak constraints could be closed by analyzing the data from other X-ray missions(CHANDRA, Suzaku) which have instrumental lines at different energies. Assuming these studies done we can achieve one of three results. At the worst case we would not find any hints for the DM decay line in the whole available to the current X-ray missions energy band. Due to already high exposure time of each mission ($\sim 5-10$~yrs) this will mean that current generation of instruments can not detect corresponding line in the observable  future. However even in this case, planning missions (LOFT, Astro-H) still could be useful for the line detection. The targets for these missions, as well as the desired parameters for the mission that could significantly improve current constraints can be estimated in this case.

    As the alternative the search within all available X-ray data could left us with one or few $\sim3\sigma$-candidate lines that have the spatial distribution agreed with DM. In this case even current instruments can provide the significant improvement in the detected significance, given the dedicated long-time observation of DM-dominated objects. Planning missions in this case would be able to detect the candidate line with relatively short exposure times.

    At the most optimistic case the analysis will reveal the DM-decay line with high $\gtrsim 5\sigma$ significance, that will give a good hint for the correctness of $\nu$MSM model and will have a significant impact on all beyond-SM physics.

\item \textbf{Quantum chiral anomaly effects in neutron stars}. In Boyarsky et.al.(1109.3350) was demonstrated that the evolution of magnetic fields in a plasma with temperature $T\gtrsim 10$~MeV is strongly affected by the quantum chiral anomaly -- an effect that has been neglected previously. In the early universe this can lead to magnetic helicity transfer from shorter to
longer scales and to the effective generation of lepton asymmetry which can strongly affect many processes. The pre-requisite for this is the small ratio of chirality- to magnetic-flipping scales ($\gamma=\Gamma_f/\Gamma_B \ll 1$). The same conditions can be also achieved in the current-Universe objects, e.g. (AGN) jets and accreation discs in which strong helical magnetic fields and very high energy electrons are present. However it can be shown that the ratio $\Gamma_f/\Gamma_B$ is strongly depends on the conductivity of these objects for which only estimates are known. Another problem with jets is not defined temperature, since the jet's particles can have non-Maxwell distribution and main fraction of their energy corresponds to the bulk motion.

In (1204.3604) Boyarsky et. al. consider the ground state of SM plasma which is believed to be homogeneous and isotropic at high electron
 temperatures. Boyarsky et. al. have shown that the finite density of lepton or baryon numbers, due to parity-violating nature of the weak
  interactions, makes the initially homogeneous ground state unstable by developing a long-range magnetic field.
  The same effects are estimated to have place in other types of objects -- neutron stars(NS). However the developed in (1204.3604) formalism could not be
   directly applied to NS, since these systems have finite volume and are not closed. The question of the impact of the generation of strong magnetic fields
    in neutron stars on their evolution and phenomenology is still open.

\item \textbf{High-energy astrophysics.} Broad topic, which covers a lot of very interesting questions -- the origin of recently discovered 130~GeV (as well as other features at 70-90 and 100-120~GeV) -- astrophysical or instrumental? Hints of variability of GC at $>10GeV$ energies at 0.5year timescale. If real -- is it consistent with ``diffuse proton''(see [1009.2630]) or other models of $\gamma$-radiation from the GC.
\end{itemize}

\end{document}
