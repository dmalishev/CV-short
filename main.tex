\documentclass{article} \oddsidemargin -0,05 cm \textwidth
16.0cm \topmargin 0pt \textheight 24cm
\usepackage{color}
\usepackage{amsmath,amssymb}
\usepackage{floatflt}
%\usepackage[colorlinks]{hyperref}
\usepackage{psfrag}
\usepackage{jcappub} %
\bibliographystyle{JHEP-2}
\newcommand{\bc}{\begin{center}}
\newcommand{\be}{\begin{equation}}
\newcommand{\ec}{\end{center}}
\newcommand{\ee}{\end{equation}}
\renewcommand\refname{}
\begin{document}
\section*{Dr. Denys Malyshev}
 \noindent
Institut f{\"u}r Astronomie und Astrophysik T{\"u}bingen, Universit{\"a}t T{\"u}bingen,  T{\"u}bingen, Germany \\
\emph{e-mail}: denys.malyshev@astro.uni-tuebingen.de \\
Date of birth: 24 July, 1984\\
Marital status: single, no children\\
\textbf{Current status:}
\begin{itemize}
\item June 2016 -- \\
Institut f{\"u}r Astronomie und Astrophysik T{\"u}bingen, Universit{\"a}t T{\"u}bingen,  T{\"u}bingen, Germany  \\
Postdoc researcher

\item July 2013 -- April 2016\\
ISDC Data Centre for astrophysics, Versoix, Switzerland \\
Postdoc researcher

\item February 2011 -- July 2013\\
Bogolyubov Institute for Theoretical Physics, Kiev, Ukraine\\
Postdoc researcher

\textbf{Education:}
 \item  October 2007- February 2011\\
    Dublin Institute for Advanced Studies, Ph.D. in Astrophysics \\
    \emph{PhD. Diploma}:``Multiwavelength properties of the sources of cosmic ray protons'' \\(supervizor:Prof. F.Aharonian)


  \item      September 2005- June 2007 \\
  Physics and Technical Institute, NTUU "KPI" and
      Scientific-Educational Center of Bogolyubov Institute for theoretical physics
      M.Sc. in Theoretical Physics.\\
           \emph{MSc. Diploma}: ``Searching of the decay line of Dark Matter with SPI/INTEGRAL.''\\(supervizor:Dr. A.Neronov)


    \item      September 2001-2005\\
      Physics and Technical Institute, NTUU "KPI" and
      Scientific-Educational Center of Bogolyubov Institute for theoretical physics
      B.Sc in Theoretical Physics\\
           \emph{BSc. Diploma}: ``Entropy of the black holes with non-trivial topology in 2+1 dimensions.''\\ (supervizor:Dr. K.Krasnov)

\end{itemize}
\subsection*{Services in National and/or International Committees}
-- Member of XMM AO-15 time allocation committee, Italy, 2015 \\
-- Member of XMM AO-11 time allocation committee, ESA, Spain, 2011 \\

%\subsection*{ Schools, conferences, meetings attended}
%-- Sources of Galactic cosmic rays, APC, Paris - December 7-9, 2016 \\
%-- 28th TEXAS Symposium on Relativistic Astrophysics, Switzerland, Geneva, 2015\\
%-- Member of XMM AO-15 time allocation committee, Italy, 2015 \\
%-- International Space Science Institute (Bern, Switzerland) team meetings on Galactic center, 2012 \\
%-- Member of XMM AO-11 time allocation committee, ESA, Spain, 2011 \\
%-- International Space Science Institute (Bern, Switzerland) team meetings on Gamma-ray loud binaries, 2010-2012 \\
%--Astronomical Science Group of Ireland (ASGI) meeting in Cork, Ireland, 2008 (talk given)\\
%--4th Heidelberg International Symposium on High Energy Gamma-Ray Astronomy, Heidelberg, Germany, 2008\\
%--Physics of Neutron Stars - 2008, St.Petersburg, Russia (poster presented)\\
%--Particle in cell (PIC) simulations of relativistic collisionless shocks, Dublin, Ireland, 2008\\
%--Les-Houches Winter School, France, 2007 ``The violent universe''\\
%--3-rd INTEGRAL Data Analysis Workshop, Switzerland, 2006\\
%--2003-2007 summer and winter Dubna schools on theoretical physics, Dubna, Russia
\newpage
\section*{Publications}
\begingroup
% \smallskip\\
{\small
  \renewcommand{\section}[2]{}
  (36 publications with 971 citations, H-index=16; collaborational papers excluded)
  % Bibliography and bibfile
\def\aj{AJ}%
          % Astronomical Journal
\def\actaa{Acta Astron.}%
          % Acta Astronomica
\def\araa{ARA\&A}%
          % Annual Review of Astron and Astrophys
\def\apj{ApJ}%
          % Astrophysical Journal
\def\apjl{ApJ}%
          % Astrophysical Journal, Letters
\def\apjs{ApJS}%
          % Astrophysical Journal, Supplement
\def\ao{Appl.~Opt.}%
          % Applied Optics
\def\apss{Ap\&SS}%
          % Astrophysics and Space Science
\def\aap{A\&A}%
          % Astronomy and Astrophysics
\def\aapr{A\&A~Rev.}%
          % Astronomy and Astrophysics Reviews
\def\aaps{A\&AS}%
          % Astronomy and Astrophysics, Supplement
\def\azh{AZh}%
          % Astronomicheskii Zhurnal
\def\baas{BAAS}%
          % Bulletin of the AAS
\def\bac{Bull. astr. Inst. Czechosl.}%
          % Bulletin of the Astronomical Institutes of Czechoslovakia
\def\caa{Chinese Astron. Astrophys.}%
          % Chinese Astronomy and Astrophysics
\def\cjaa{Chinese J. Astron. Astrophys.}%
          % Chinese Journal of Astronomy and Astrophysics
\def\icarus{Icarus}%
          % Icarus
\def\jcap{J. Cosmology Astropart. Phys.}%
          % Journal of Cosmology and Astroparticle Physics
\def\jrasc{JRASC}%
          % Journal of the RAS of Canada
\def\mnras{MNRAS}%
          % Monthly Notices of the RAS
\def\memras{MmRAS}%
          % Memoirs of the RAS
\def\na{New A}%
          % New Astronomy
\def\nar{New A Rev.}%
          % New Astronomy Review
\def\pasa{PASA}%
          % Publications of the Astron. Soc. of Australia
\def\pra{Phys.~Rev.~A}%
          % Physical Review A: General Physics
\def\prb{Phys.~Rev.~B}%
          % Physical Review B: Solid State
\def\prc{Phys.~Rev.~C}%
          % Physical Review C
\def\prd{Phys.~Rev.~D}%
          % Physical Review D
\def\pre{Phys.~Rev.~E}%
          % Physical Review E
\def\prl{Phys.~Rev.~Lett.}%
          % Physical Review Letters
\def\pasp{PASP}%
          % Publications of the ASP
\def\pasj{PASJ}%
          % Publications of the ASJ
\def\qjras{QJRAS}%
          % Quarterly Journal of the RAS
\def\rmxaa{Rev. Mexicana Astron. Astrofis.}%
          % Revista Mexicana de Astronomia y Astrofisica
\def\skytel{S\&T}%
          % Sky and Telescope
\def\solphys{Sol.~Phys.}%
          % Solar Physics
\def\sovast{Soviet~Ast.}%
          % Soviet Astronomy
\def\ssr{Space~Sci.~Rev.}%
          % Space Science Reviews
\def\zap{ZAp}%
          % Zeitschrift fuer Astrophysik
\def\nat{Nature}%
          % Nature
\def\iaucirc{IAU~Circ.}%
          % IAU Cirulars
\def\aplett{Astrophys.~Lett.}%
          % Astrophysics Letters
\def\apspr{Astrophys.~Space~Phys.~Res.}%
          % Astrophysics Space Physics Research
\def\bain{Bull.~Astron.~Inst.~Netherlands}%
          % Bulletin Astronomical Institute of the Netherlands
\def\fcp{Fund.~Cosmic~Phys.}%
          % Fundamental Cosmic Physics
\def\gca{Geochim.~Cosmochim.~Acta}%
          % Geochimica Cosmochimica Acta
\def\grl{Geophys.~Res.~Lett.}%
          % Geophysics Research Letters
\def\jcp{J.~Chem.~Phys.}%
          % Journal of Chemical Physics
\def\jgr{J.~Geophys.~Res.}%
          % Journal of Geophysics Research
\def\jqsrt{J.~Quant.~Spec.~Radiat.~Transf.}%
          % Journal of Quantitiative Spectroscopy and Radiative Trasfer
\def\memsai{Mem.~Soc.~Astron.~Italiana}%
          % Mem. Societa Astronomica Italiana
\def\nphysa{Nucl.~Phys.~A}%
          % Nuclear Physics A
\def\physrep{Phys.~Rep.}%
          % Physics Reports
\def\physscr{Phys.~Scr}%
          % Physica Scripta
\def\planss{Planet.~Space~Sci.}%
          % Planetary Space Science
\def\procspie{Proc.~SPIE}%
          % Proceedings of the SPIE
\let\astap=\aap
\let\apjlett=\apjl
\let\apjsupp=\apjs
\let\applopt=\ao
 \nocite{*}
  \bibliography{cv}
}
\endgroup
\vspace{1cm}
please see the full list at http://goo.gl/lVCW5Z
\end{document}
