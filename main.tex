\documentclass{article} \oddsidemargin -0,05 cm \textwidth
16.0cm \topmargin 0pt \textheight 24cm
\usepackage{color}
\usepackage{amsmath,amssymb}
\usepackage{floatflt}
%\usepackage[colorlinks]{hyperref}
\usepackage{psfrag}
\usepackage{jcappub} %
\bibliographystyle{JHEP-2}
\newcommand{\bc}{\begin{center}}
\newcommand{\be}{\begin{equation}}
\newcommand{\ec}{\end{center}}
\newcommand{\ee}{\end{equation}}
\renewcommand\refname{}
\begin{document}
\section*{Dr. Denys Malyshev}
 \noindent
Institut f{\"u}r Astronomie und Astrophysik T{\"u}bingen, Universit{\"a}t T{\"u}bingen,  T{\"u}bingen, Germany \\
\emph{e-mail}: denys.malyshev@astro.uni-tuebingen.de \\
Date of birth: 24 July, 1984\\
Marital status: single, no children\\
\textbf{Current status:}
\begin{itemize}
\item June 2016 -- \\
Institut f{\"u}r Astronomie und Astrophysik T{\"u}bingen, Universit{\"a}t T{\"u}bingen,  T{\"u}bingen, Germany  \\
Postdoc researcher

\item July 2013 -- April 2016\\
ISDC Data Centre for astrophysics, Versoix, Switzerland \\
Postdoc researcher

\item February 2011 -- July 2013\\
Bogolyubov Institute for Theoretical Physics, Kiev, Ukraine\\
Postdoc researcher

\textbf{Education:}
 \item  October 2007- February 2011\\
    Dublin Institute for Advanced Studies, Ph.D. in Astrophysics \\
    \emph{PhD. Diploma}:``Multiwavelength properties of the sources of cosmic ray protons'' \\(supervizor:Prof. F.Aharonian)


  \item      September 2005- June 2007 \\
  Physics and Technical Institute, NTUU "KPI" and
      Scientific-Educational Center of Bogolyubov Institute for theoretical physics
      M.Sc. in Theoretical Physics.\\
           \emph{MSc. Diploma}: ``Searching of the decay line of Dark Matter with SPI/INTEGRAL.''\\(supervizor:Dr. A.Neronov)


    \item      September 2001-2005\\
      Physics and Technical Institute, NTUU "KPI" and
      Scientific-Educational Center of Bogolyubov Institute for theoretical physics
      B.Sc in Theoretical Physics\\
           \emph{BSc. Diploma}: ``Entropy of the black holes with non-trivial topology in 2+1 dimensions.''\\ (supervizor:Dr. K.Krasnov)

\end{itemize}
\subsection*{Services in National and/or International Committees}
-- Member of XMM AO-15 time allocation committee, Italy, 2015 \\
-- Member of XMM AO-11 time allocation committee, ESA, Spain, 2011 \\

%\subsection*{ Schools, conferences, meetings attended}
%-- Sources of Galactic cosmic rays, APC, Paris - December 7-9, 2016 \\
%-- 28th TEXAS Symposium on Relativistic Astrophysics, Switzerland, Geneva, 2015\\
%-- Member of XMM AO-15 time allocation committee, Italy, 2015 \\
%-- International Space Science Institute (Bern, Switzerland) team meetings on Galactic center, 2012 \\
%-- Member of XMM AO-11 time allocation committee, ESA, Spain, 2011 \\
%-- International Space Science Institute (Bern, Switzerland) team meetings on Gamma-ray loud binaries, 2010-2012 \\
%--Astronomical Science Group of Ireland (ASGI) meeting in Cork, Ireland, 2008 (talk given)\\
%--4th Heidelberg International Symposium on High Energy Gamma-Ray Astronomy, Heidelberg, Germany, 2008\\
%--Physics of Neutron Stars - 2008, St.Petersburg, Russia (poster presented)\\
%--Particle in cell (PIC) simulations of relativistic collisionless shocks, Dublin, Ireland, 2008\\
%--Les-Houches Winter School, France, 2007 ``The violent universe''\\
%--3-rd INTEGRAL Data Analysis Workshop, Switzerland, 2006\\
%--2003-2007 summer and winter Dubna schools on theoretical physics, Dubna, Russia
\newpage
\section*{Publications}
\begingroup
% \smallskip\\
{\small
  \renewcommand{\section}[2]{}
  (36 publications with 971 citations, H-index=16; collaborational papers excluded)
  \input{journals.tex}
 \nocite{*}
  \bibliography{cv}
}
\endgroup
\vspace{1cm}
please see the full list at http://goo.gl/lVCW5Z
\end{document}
